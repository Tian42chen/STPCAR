% CVPR 2023 Paper Template
% based on the CVPR template provided by Ming-Ming Cheng (https://github.com/MCG-NKU/CVPR_Template)
% modified and extended by Stefan Roth (stefan.roth@NOSPAMtu-darmstadt.de)

\documentclass{source/Paper}

% Support for easy cross-referencing

%%%%%%%%% PAPER ID  - PLEASE UPDATE
\def\cvprPaperID{*****} % *** Enter the CVPR Paper ID here
\def\confName{CVPR}
\def\confYear{2023}


\begin{document}

%%%%%%%%% TITLE - PLEASE UPDATE
\title{\LaTeX\ Author Guidelines for \confName~Proceedings}

\author{First Author\\
Institution1\\
Institution1 address\\
{\tt\small firstauthor@i1.org}
% For a paper whose authors are all at the same institution,
% omit the following lines up until the closing ``}''.
% Additional authors and addresses can be added with ``\and'',
% just like the second author.
% To save space, use either the email address or home page, not both
\and
Second Author\\
Institution2\\
First line of institution2 address\\
{\tt\small secondauthor@i2.org}
}
\maketitle

%%%%%%%%% ABSTRACT
\begin{abstract}
    The ABSTRACT is to be in fully justified italicized text, at the top of the left-hand column, below the author and affiliation information.
    Use the word ``Abstract'' as the title, in 12-point Times, boldface type, centered relative to the column, initially capitalized.
    The abstract is to be in 10-point, single-spaced type.
    Leave two blank lines after the Abstract, then begin the main text.
    Look at previous CVPR abstracts to get a feel for style and length.
\end{abstract}

% \begin{enumerate}
%   \item 简介
%   \item 最近工作 (列举相关即可)
%   \item 方法
%   \item 实验 (可视化)
%   \subitem 消融实验
%   \item 结论
% \end{enumerate}
% 甚至可以在首页右上角加一张图片

%%%%%%%%% BODY TEXT
\input{content/Introduction.tex}
\input{content/Related Work.tex}
\input{content/Method.tex}
\section{Experiments}
\label{sec:exp}


\subsection{Ablation Study}
\input{content/Conclusion.tex}

\cite{Authors14}

%%%%%%%%% REFERENCES
{\small
\bibliographystyle{reference/ieee_fullname}
\bibliography{reference/egbib}
}

\end{document}
